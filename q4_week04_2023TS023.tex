\documentclass[11pt]{jsarticle}

\title{宮澤研究室 文献紹介}
\author{2023TS023:可児桜将}
\date{2025年12月14日}

\begin{document}

\maketitle

\section{今回読んだ文献について}
今回私は情報処理学会の電子図書館から華山 宣胤氏、野上 竜一が書いた「『ゲーム制作総合演習』による教育事例報告」という文献を読んだ。
\subsection{概要}
本報告書は、尚美学園大学芸術情報学部情報表現学科において2009年より実施されている「ゲーム制作総合演習」の教育実践について報告したものである。

本授業は、TVゲーム開発を題材に、マーケティングリサーチ、プログラミング、プレゼンテーションを統合した総合演習として構成されている。

目的は、単に与えられた使用に従ってシステムを構成する技術者ではなく、商品コンセプトの設定から実装までを一貫して担える人材の育成である。

\section{授業構成}

本演習は、以下の3段階で進められる。

\subsection{マーケティング・リサーチによるコンセプト設定}

まず、既存のTVゲームを競合として選定し、「アクション性」「育成要素」「ストーリー性」などの評価項目を設定する。
この分析結果に基づき、市場に存在しない新しいゲームコンセプトを設定する。

\subsection{Processingによるゲーム制作}

設定されたコンセプトに基づき、Processing言語を用いてゲーム制作を行う。
授業時間の制約を考慮し、予め用意された複数のゲーム雛形を利用することで、短期間でもコンセプトを反映したゲーム開発を可能としている。

\subsection{プレゼンテーションと評価}

完成したゲーブとコンセプトについてスライドを用いたプレゼンテーションを行う。
評価は教員だけでなく履修者同士の相互評価によっても行われ、企画力、分析力、表現力を含めた総合的な評価が行われる。

\section{教育的効果とまとめ}
本演習は、グループ作業を通して役割分担や工程管理を学ばせるとともに、「商品コンセプト設定」「コンテンツ制作」「プレゼンテーション」という本学科の多くの専門科目に共通する能力を育成する点に特徴がある。

本報告では、本演習が実践的なゲーム開発教育として有効であることが示されており、今後は他の専門科目への展開も期待されている。

\end{document}
